\chapter{To-be named Chapter}
\label{sec:implementation}

\section{Study Area}
The study area encompasses the geographic mainland region of Portugal, extending 89,000 km\textsuperscript{2} located between latitude 42.3°N to 36.7°N and longitude 9.8°W to 6.0°W. Portugal experiences a Mediterranean climate influenced by the Atlantic Ocean and characterized by a mostly wet and cool season followed by a dry summer \cite{Mora2020, portugalClimate01, Marques2011}.

Altitudes range from sea level to  2000 metres, and higher elevations are more prevelent central and northern regions \cite{Marques2011}.

The northern region is characterised by lower temperatures and higher values of precipitation than the southern region \cite{Marques2011}. The average annual temperature ranges from 7 to 18°C, while the annual rainfall ranges from 400 to 2,800 mm.

Portugal's vegetation is a blend of Atlantic, European, Mediterranean, and African species \cite{britannica_portugal_climate}, and four tree species account for 80\% of all the forest area:\textit{Pinus pinaster}, \textit{Eucalyptus globulus}, \textit{Quercus suber}, and \textit{Quercus rotundifolia} \cite{Marques2011}.





\section{Dataset Sources}

\subsection{Wildfire Occurrences}
Historical records of fire occurrences were taken from \cite{centraldedados_incendios_website, centraldedados_incendios} and \cite{icnf2024}. The records featured in \cite{centraldedados_incendios_website, centraldedados_incendios} span across 35 years, from 1980 to 2015, and hold 791453 instances of wildfires. Its most relevant features are encapsulated in table \ref{historical_occurrences_1980_2015}, and table \ref{number_occurrences_1980_2013} holds the fire records for each year.


The records from \cite{icnf2024} span across 10 years, from 2013 to 2023, and were retrieved with an API \href{https://fogos.icnf.pt/localizador/webserviceocorrencias.asp?ano=YEAR}{endpoint}. It features 140514 entries, and its most important fields are described in table \ref{number_occurrences_1980_2013}, while table \ref{number_occurrences_2013_2023} holds the number of occurrences individually by year.

\begin{table}[H]
	\caption{Field description of Historical fires from 1980 to 2015 \cite{centraldedados_incendios_website, centraldedados_incendios}}
	\label{historical_occurrences_1980_2015}
	\centering
	\small
	\begin{tabular}{cp{7.5cm}} % Adjust width as necessary
		\hline
		\textbf{Variable} & \textbf{Description}\\
		\hline
		ano  & Year of fire occurrence \\
		codigo\_sgif & Unique identifier for the fire occurrence \\
		tipo  & Kind of wildfire. The available options are: forest, slash-and-burn, false alarm, and agricultural.\\
		distrito  & District of fire occurrence. \\
		concelho  & Municipality of fire occurrence. \\
		freguesia  & Parish of fire occurrence.\\
		local  & Location of fire occurrence. \\
		ine  & Not described or explained anywhere. \\
		x | y  & Wildfire location. \\
		data\_alerta  & Wildfire warning date. \\
		hora\_alerta  & Wildfire warning hour. \\
		data\_extincao & Wildfire complete extinguishment date. \\
		hora\_extincao & Hour of wildfire complete extinguishment. \\
		data\_primeira\_intervencao & Date of first fire intervention \\
		hora\_primeira\_intervencao & Hour of first fire intervention \\
		fonte\_alerta & Authority or group of people who reported the fire first. \\
		nut & A Unique identifier for a given nomenclature of territorial units for statistics. \\
		area\_povoamento & Burnt settlement area. \\
		area\_mato & Burnt bush area.  \\
		area\_agricola & Burnt agricultural area. \\
		area\_pov\_mato & Sum of burned area from the burnt settlement area and burnt agricultural area.\\
		area\_total & Total burnt area. \\
		reacendimento & Describes if a given fire is a re-ignition from a previous wildfire. \\
		queimada & Identifies if a fire is a slash-and-burn. \\
		falso\_alarme & Identifies if it is a false alarm. \\
		fogacho & Identifies if it is a specific type of fire named a blaze. \\
		incendio & Identifies if it is a fire. \\
		causa & Numerical identifier for the fire cause. \\
		tipo\_causa & Description of fire cause. The available options are unknown, deliberate, natural, negligent fire, and undefined. \\
		\hline
	\end{tabular}
\end{table}


\begin{table}[H]
	\caption{Number of Fire occurrences between 1980 and 2015}
	\label{number_occurrences_1980_2013}
	\centering
	\begin{tabular}{lc}
		\hline
		Year & \multicolumn{1}{l}{Number of Occurrences} \\ \hline
		1980 & 2346                                      \\
		1981 & 6727                                      \\
		1982 & 3625                                      \\
		1983 & 4536                                      \\
		1984 & 7355                                      \\
		1985 & 8439                                      \\
		1986 & 5036                                      \\
		1987 & 7703                                      \\
		1988 & 6130                                      \\
		1989 & 21895                                     \\
		1990 & 10743                                     \\
		1991 & 14327                                     \\
		1992 & 14951                                     \\
		1993 & 14799                                     \\
		1994 & 19983                                     \\
		1995 & 31116                                     \\
		1996 & 28626                                     \\
		1997 & 23494                                     \\
		1998 & 34675                                     \\
		1999 & 25473                                     \\
		2000 & 34107                                     \\
		2001 & 31582                                     \\
		2002 & 33697                                     \\
		2003 & 30345                                     \\
		2004 & 34722                                     \\
		2005 & 50364                                     \\
		2006 & 31445                                     \\
		2007 & 31122                                     \\
		2008 & 23139                                     \\
		2009 & 34979                                     \\
		2010 & 32357                                     \\
		2011 & 35941                                     \\
		2012 & 30740                                     \\
		2013 & 27372                                     \\
		2014 & 11387                                     \\
		2015 & 23175                                     \\ \hline
		Total & 791453                                   \\
		\hline
	\end{tabular}
\end{table}


\begin{table}[H]
	\caption{Field description of Historical fires from 2013 to 2023 \cite{icnf2024}}
	\label{historical_occurrences_2013_2023}
	\centering
	\small
	\begin{tabular}{cp{8.5cm}}
		\hline
		\textbf{Variable} & \textbf{Description}\\
		\hline
		CODIGO id  & Unique identifier for the fire occurrence \\
		DISTRITO  & District of fire occurrence. \\
		TIPO  & Kind of wildfire. The available options are: forest and agricultural fire.\\
		ANO  & Year of fire occurrence \\
		AREAPOV & Burnt settlement area. \\
		AREAMATO & Burnt bush area.  \\
		AREAAGRIC & Burnt agricultural area. \\
		AREATOTAL & Total burnt area. \\
		REACENDIMENTOS & Boolean value for reignition. \\
		FOGACHO & Boolean value for small fire. \\
		Incendio & Boolean value for fire. \\
		Agricola & Boolean value for agricultural fire. \\
		DATAALERTA & Wildfire warning date. \\
		HORAALERTA & Wildfire warning hour. \\
		LOCAL & Location of fire occurrence \\
		CONCELHO & Municipality of fire occurrence. \\
		FREGUESIA & Parish of fire occurrence\\
		FONTEALERTA & Authority or group of people who reported the fire first. \\
		DIA & Day of the fire occurrence. \\
		MES & Month of fire occurrence. \\
		HORA & Fire hour of occurrence. \\
		CAUSA & Numerical identifier for the fire cause. \\
		TIPOCAUSA & Description of fire cause. The available options differ from year to year. \\
		DHINICIO & Firefighting starting date and time \\
		DHFIM & Ending of firefighting efforts\\
		DURACAO & Number of minutes it took to extinguish the fire \\
		HAHORA & Not described \\
		DATAEXTINCAO & Fire extinguish date \\
		HORAEXTINCAO & Fire extinguish hour \\
		QUEIMA & Boolean value for intentional small fire\\
		LAT & Latitude coordinate of fire occurrence\\
		LON & Longitude coordinate of fire occurrence\\
		TEMPERATURA & Value for temperature at the time of fire occurrence\\
		HUMIDADERELATIVA & Relative Humidity value\\
		VENTOINTENSIDADE & Wind velocity\\
		VENTODIRECAO\_VETOR & Direction of wind\\
		PRECIPITACAO & Value for rainfall\\
		FFMC & Fine fuel moisture code\\
		DMC & Duff moisture code\\
		DC & Drought code\\
		ISI & Initial Spread index\\
		BUI & Build Up Index\\
		FWI & Fire Weather Index\\
		DSR & Daily Severity Rating\\
		ALTITUDEMEDIA & Mean altitude\\
		DECLIVEMEDIO & Mean Slope\\
		\hline
	\end{tabular}
\end{table}

\begin{table}[H]
	\centering
	\caption{Number of Fire occurrences between 2013 and 2023}
	\label{number_occurrences_2013_2023}
	\begin{tabular}{lc}
		Year  & \multicolumn{1}{l}{Number of Occurrences} \\ \hline
		2013  & 24479                                     \\
		2014  & 10286                                     \\
		2015  & 19669                                     \\
		2016  & 16131                                     \\
		2017  & 21074                                     \\
		2018  & 12296                                     \\
		2019  & 10909                                     \\
		2020  & 9713                                      \\
		2021  & 6996                                      \\
		2022  & 1335                                      \\
		2023  & 7626                                      \\ \hline
		Total & 140514                                    \\ \hline
	\end{tabular}
\end{table}



\subsection{Weather Variables}
Historical weather variables were obtained via the API \cite{Zippenfenig_Open-Meteo}. This includes daily and hourly weather data extraction from all across the world. The API compares the provided location to several reanalysis datasets and returns the most optimal result based on location. It includes a full database of historical weather conditions dating back to 1940 from multiple sources (table \ref{Datasets_api}). 

It incorporates observations from weather stations, aeroplanes, buoys, radar, and satellites and fills gaps in data using mathematical models to estimate the values of different weather variables \cite{Zippenfenig_Open-Meteo}.


The variables extracted from \textit{Open-Meteo} are on table \ref{other_variables} as well as the unit for each weather variable.


\begin{table}[H]
	\caption{API Weather Sources \cite{Zippenfenig_Open-Meteo, Hersbach_ERA5, Munoz_ERA5_LAND, Schimanke_CERRA}}
	\centering
	\label{Datasets_api}
	\begin{tabular}{llll}
		\hline
		\multicolumn{1}{c}{Dataset} & \multicolumn{1}{c}{Region} & \multicolumn{1}{c}{Resolution} & \multicolumn{1}{c}{Availability} \\ \hline
		ECMFWF IFS & Global & 9km, Hourly  & 2017 to present   \\
		ERA5       & Global & 25km, Hourly & 1940 to present   \\
		ERA5-Land  & Global & 11km, Hourly & 1950 to present   \\
		CERRA      & Europe & 5km, Hourly  & 1985 to June 2021
	\end{tabular}
\end{table}




\subsection{Fuel Variables}
Fuel variables encompass two sources: Copernicus Climate Change Service \cite{CopernicusCDS2019} and Forestry Inventory 2015 \cite{uva2021forestry,https://doi.org/10.15468/dl.zwfmbt}

The Copernicus Climate Change Service contains historical reconstructions of fire danger indices, i.e., variables that emphasise conditions suitable for the origin, spread, and sustainability of naturally occurring fires. The fire danger indices are obtained from historical simulations and weather forecasts, and the available data starts in January 1940 and extends all the way through 2023. Variables contained in the dataset are expressed in the table \ref{copernicus_danger_indices}.

The Forestry Inventory 2015 contains 579422 occurrences of forest species on mainland Portugal. The data was gathered using aerial images and ground surveys. The most important features are described in table \ref{forest_inventory}.


\begin{table}[H]
	\caption{Fire danger indices from historical data \cite{CopernicusCDS2019}}
	\label{copernicus_danger_indices}
	\centering
	\small
	\begin{tabular}{ccp{7.5cm}} % Adjust width as necessary
		\hline
		\textbf{Variable} & \textbf{Unit} & \textbf{Description}\\
		\hline
		Build-up index & Dimensionless & Weighted combination of the Duff moisture code and Drought code. \\
		\hline
		Burning index & Dimensionless & Measure that explains how difficult it is to control a fire. \\
		\hline
		Danger rating & Dimensionless & Equivalent to the FWI but with class level definitions of very low, low, medium, high, very high and extreme. \\
		\hline
		Drought code & Dimensionless & Component representing fuel availability, and the influence of recent temperatures and rainfall events on fuel availability. \\
		\hline
		Duff moisture code & Dimensionless & Moisture content in loosely-compacted organic layers of moderate depth. Duff moisture code fuels are affected by rain, temperature and relative humidity. \\
		\hline
		Energy release component & $J/m^2$ & Available energy within the burning front at the head of a fire. \\
		\hline
		Fine fuel moisture code & Dimensionless &  Moisture content in litter. Representative of the top litter layer less than 1-2 cm deep. \\
		\hline
		Fire daily severity index & Dimensionless &  A numerical assessment of the difficulty of controlling flames. \\
		\hline
		Fire danger index & Dimensionless &  Metric that hold the chances of a fire starting. \\
		\hline
		Fire weather index & Dimensionless &  Combination of Initial spread index and Build-up index. Numerical rating of the potential fire intensity. \\
		\hline
		Ignition component & \% &  Probability of a firebrand that will require suppression action. \\
		\hline
		Keetch-Byram drought index & Dimensionless &  The total impact of evapotranspiration and precipitation in causing cumulative moisture shortage in deep duff and higher soil layers. \\
		\hline
		Spread component & Dimensionless &  Measure of the speed at which a headfire will spread. \\
		\hline
	\end{tabular}
\end{table}


\begin{table}[H]
	\caption{Forest Inventory 2015}
	\label{forest_inventory}
	\centering
	\small
	\begin{tabular}{cp{7.5cm}} % Adjust width as necessary
		\hline
		\textbf{Variable} & \textbf{Description}\\
		\hline
		gbifID | datasetKey | ocurrenceID  & Identifiers for the occurrence of trees and the dataset. \\
		kingdom & Kingdom classification of a given Tree. \\
		phylum & Phylum classification of a given Tree. \\
		class & Taxonomic class. \\
		order & Taxonomic Order of a Tree. \\
		genus & Tree genus. \\
		species & Data containing the species of a given tree. \\
		taxonRank & Data containing the highest taxonomic rank available for a given tree group. \\
		scientificName | verbatimScientificName & Scientific name for the available taxonomic classification. \\
		
		verbatimScientificNameAuthorship & Scientific name authorship for the available taxonomic classification. \\
		
		countryCode & Contry code of Portugal. \\
		
		locality & Name of a locality containing a given tree. \\
		
		stateProvince &  Name of a district containing a given tree. \\
		
		occurrenceStatus & Describes if a tree is still present.   \\
		
		decimalLatitude & Latitude for the tree occurrence. \\
		
		decimalLongitude & Longitude for the tree occurrence \\
		
		coordinateUncertaintyInMeters & Uncertainty for a given tree location in metres. \\
		
		eventDate | year & Year of event record. \\
		
		taxonKey & Taxonomic key for the highest available classification for a tree \\
		
		speciesKey & Individual key for a given tree species if available. \\
		
		speciesKey & Individual key for a given tree species if available. \\
		
		institutionCode & Unique identifier for ICNF. \\
		
		collectionCode & Unique collection identifier for the institutionCode. \\
	\end{tabular}
\end{table}


\subsection{Topography Variables}
The sources of topography variables and brief description of them

\cite{rasterio}

\cite{Danielson2011} Terrain Elevation Data from around the globe



\begin{table}[H]
	
	\centering
	\caption{Topography, Weather and Fuel variables}
	\label{other_variables}
	\begin{tabular}{cccc}
		\hline
		\textbf{Variable Type} &
		\textbf{Variable Name} &
		\textbf{Resolution} \\ \hline
		Topography &
		\begin{tabular}[c]{@{}c@{}}Elevation (m)\\ Slope (0-90)\\ Topographic roughness Index\end{tabular} &
		\_,7.5 arc-seconds, \_ \\ \hline
		Weather &
		\begin{tabular}[c]{@{}c@{}}Temperature (°C)\\ Relative Humidity (\%)\\ Dew (°C)\\ Apparent Temperature (°C)\\ Pressure (hPa)\\ Surface Pressure (hPa)\\ Precipitation (mm)\\ Rain (mm)\\ Snowfall (cm)\\ Cloud Cover Low/Mid/High (\%)\\ Shortwave Radiation ($W/m^2$)\\ Direct Radiation ($W/m^2$)\\ Direct Normal Irradiance ($W/m^2$)\\ Diffuse Radiation ($W/m^2$)\\ Global Tilted Irradiance ($W/m^2$)\\ Sunshine Duration (Seconds)\\ Wind Speed 10m/100m (km/h)\\ Wind Direction 10m/100m (°)\\ Wind Gusts 10m (km/h)\\ Et0 Evapotranspiration (mm)\\ Weather Code (WMO code)\\ Snow Depth (metres)\\ Vapour Pressure Deficit (kPa)\\ Soil Temperature 0 to 255cm (°C)\\ Soil Moisture 0 to 255cm ($m^3/m^3$)\end{tabular}
		&
		0.25/0.1/9km/5km, \_ , Hourly \\ \hline
		Fuel &
		\begin{tabular}[c]{@{}c@{}}Fine fuel moisture code\\ Duff moisture code\\ Drought code\\ Initial Spread index \\ Build Up Index \\ Fire weather index\\ Tree Species\end{tabular} &
		\begin{tabular}[c]{@{}c@{}}0.25 , \_ , Daily\\ \\ \\ \\ \\ -\end{tabular} \\ \hline
	\end{tabular}
\end{table}


\section{Data Preparation}


























\section{Variables Explanation}
Slope has a significant influence on fire behaviour since it speeds its spread \cite{Marques2011}.









\subsection{Open-meteo hourly weather variables \cite{Zippenfenig_Open-Meteo}}
Historical Weather \cite{Zippenfenig_Open-Meteo} from \cite{Hersbach_ERA5}, \cite{Munoz_ERA5_LAND} and \cite{Schimanke_CERRA}
\label{open_meteo}
\begin{table}[h!]
\caption{Hourly weather variables from Open-meteo}
\centering
\small
\begin{tabular}{|c|c|p{10cm}|} % Adjust width as necessary
\hline
\textbf{Variable} & \textbf{Unit} & \textbf{Description}\\
\hline
Temperature & °C & Air temperature 2 metres above ground. \\
\hline
Relative Humidity & \% & Relative humidity 2 metres above ground. \\
\hline
Dew & °C & Dew point 2 metres above ground. \\
\hline
Apparent Temperature & °C & Apparent temperature is the 
result of a wind chill factor, relative humidity, and solar radiation. \\
\hline
Pressure & hPa & Atmospheric air pressure reduced to mean sea level. \\
\hline
Surface Pressure & hPa & Surface pressure reduced to mean sea level. \\
\hline
Precipitation & mm & Sum of preceding hour precipitation including rain, showers, and snow. \\
\hline
Rain & mm & Preceding hour of liquid precipitation. \\
\hline
Snowfall & cm & Preceding hour of snowfall amount. \\
\hline
Cloud cover low & \% & Fog and low level clouds up to an altitude of 2 kilometres. \\
\hline
Cloud cover mid & \% & Clouds floating at a medium level with altitudes ranging from 2 kilometres to six kilometres. \\
\hline
Cloud cover high & \% & Clouds floating at an altitude of 6 kilometres. \\
\hline
Shortwave radiation & $W/m^2$ & Shortwave solar radiation.  \\
\hline
Direct radiation & $W/m^2$ & Direct solar radiation. \\
\hline
Direct normal irradiance & $W/m^2$ & Direct solar irradiance.  \\
\hline
Diffuse radiation & $W/m^2$ & Diffuse solar radiation.  \\
\hline
Global tilted irradiance & $W/m^2$ & Total radiation received on a tilted pane.  \\
\hline
Sunshine duration & Seconds & Duration of sunshine in seconds.  \\
\hline
Wind speed at 10m & km/h & Speed of the wind, 10 metres above ground.  \\
\hline
Wind speed at 100m & km/h & Speed of the wind, 100 metres above ground.  \\
\hline
Wind direction at 10m & ° & Wind direction at 10 metres above ground.  \\
\hline
Wind direction at 100m & ° & Wind direction at 100 metres above ground.  \\
\hline
Wind gusts & km/h & Wind gusts at 10 metres above ground.  \\
\hline
Evapotranspiration & mm & Evapotranspiration value for the required irrigation for plants calculated from temperature, wind speed, humidity, and solar radiation. \\
\hline
Weather code & WMO code & Numeric codes for weather conditions.  \\
\hline
Snow depth & meters & The depth of snow on the ground.  \\
\hline
Vapour pressure deficit & kPa & Vapour pressure deficit in kilopascal.\\
\hline
Soil temperature & °C & Average soil temperature ranging from 0 to 7cm, 7 to 28cm, 28 to 100cm, and 100 to 255cm below ground.\\
\hline
Soil moisture & $m^3/m^3$ & Average soil moisture ranging from 0 to 7cm, 7 to 28cm, 28 to 100cm, and 100 to 255cm depths.\\
\hline
\end{tabular}
\end{table}



\section{Additional sources of Data}
The python library geopy \cite{geopy} was used to geolocate multiple locations, resolving district, parish, municipalities, and localities to sets of coordinates. Geopy utilises multiple geocoding web services like OpenStreetMap Nominatim and Google Geocoding API to resolve locations. 


The Google Maps service \cite{googlexmaps} was used to manually check if the extracted data from Open-Meteo corresponded to the intended location. It was also used to analyse some errors that were found in the location of some entries.




\section{Creating the dataset}
The dataset described in \ref{historical_sites_no_coords} is composed of multiple files describing historical occurrences since 1980 until 2015. Prior to 2001, the fields from each file became unstandardized, and there's no explicit parameter mentioning a natural wildfire cause. Therefore, the time frame considered was from 2001 to 2012. The latter years were rejected due to the fact that entries from \ref{historical_sites_no_coords} do not contain any explicit latitude and longitude. They rely on territorial entities such as districts, municipalities, parishes, and NUTS to describe locations.

The second historical wildfire dataset \ref{dataset_icnf2013_2023} is also composed of multiple files. Its time frame is from 2013 until 2023. Unlike dataset \ref{historical_sites_no_coords}, entries do contain an explicit latitude and longitude values. It also features descriptive territorial entities. 


\subsection{Entry selection}
Entries whose cause was deliberate or negligent fire were excluded. The fire causes contained, by order of importance, in the dataset are: natural, reignition, unknown, and undefined.
Entries that were undefined as causes differed from those with unknown causes because their cause field was blank, and entries that had unknown causes were explicitly described as unknown.

falta: tabela com o número de entradas antes e depois



\subsection{Geocoding places from 2001 to 2012 historical wildfire locations}
\label{geocoding_historical}
The dataset entries featured in \ref{historical_sites_no_coords} contain no direct field leading up to the real site location coordinates. To tackle this issue, an algorithm with the help of the geopy library \cite{geopy} was made to resolve the names of historial wildfire places to a set of coordinates.


Using multiple combinations, attempts were made to geocode the location, featuring the combinations in the table \ref{geocoding_entries_2001_2012}. The district, municipality, parish, and local (if available) of each entry were utilized for this purpose. Sometimes, the name of the exact wildfire locality was enclosed in brackets, requiring processing using strings to extract it.



\begin{table}[h!]
	\caption{Combinations for local geocoding}
	\label{geocoding_entries_2001_2012}
	\centering
	\small
	\begin{tabular}{|c|p{7.5cm}|} % Adjust width as necessary
		\hline
		\textbf{Combination}\\
		\hline
		Local, District\\
		\hline
		Local, Parish, District\\
		\hline
		Local, Parish, Municipality\\
		\hline
		Local, Parish, Municipality, District\\
		\hline
		Parish, Municipality, District\\
		\hline
		Local, Parish, District\\
		\hline
	\end{tabular}
\end{table}


These combinations caused errors in the location of some entries because the geocoders returned coordinates in other countries, such as Spain and Brazil, due to similar names in some locations. The entries that produced errors underwent recalculation, with the addition of "Portugal" at the end. An example of this usage is  {\it Parish, District, Portugal}.

After each entry was resolved, their latitude and longitude were added as values in the columns LAT and LON of each corresponding file.


A very minor sample of entries couldn't be geocoded using this method. Therefore they were manually geocoded from the Google Maps service. 



\subsection{Retrieving historical meteorological data}
In order to retrieve historical meteorological data, a Python script was made. It went through each historical fire location and downloaded weather data about the entire day regarding the wildfire. The weather data contained all the fields described in \ref{open_meteo}. 



\subsection{Linking historical wildfires with historical weather data}
Dizer número de instâncias
Every unit from each field was specified in the retrieved data, so it had to be removed.
Dizer que NA ficou com NC no dataset.




\subsection{Matching each historical wildfire with tree species}
\label{tree_species_wildfires}


O que mudei no dataset inicial das trees
dfTreesDRP['stateProvince'] = dfTreesDRP['stateProvince'].replace('Bragança District', 'Bragança')
dfTreesDRP['locality'] = dfTreesDRP['locality'].replace('Ovadas e Panchora', 'Ovadas e Panchorra')
Continha um erro, Ovadas e Panchora não existe.


variáveis das trees que foram usadas: scientificName,locality,stateProvince,occurrenceStatus,individualCount,decimalLatitude,decimalLongitude,coordinateUncertaintyInMeters,coordinatePrecision,elevation,elevationAccuracy,depth,depthAccuracy


haversine formula da distncia \cite{esri2024}.


Multiple tests were made for distance, 120metres, 500 metres.

Especies que estavam na mesma freguesia ou concelho foram associadas sem fazer cálculo de distancia. Para combater espécies duplicadas, só se adicionava uma espécie se esta não estivesse contida na entrada do fogo.
Distrito, usava-se uma distância de 1000 metros.

As restantes entradas que não obtiveram correspondencia com os outros métodos anteriores, foi feita uma análise das espécies que estavam mais perto, neste passo foram detetados erro, alguns valores tabulados do icnf não correspondiam à realidade, e alguns valores em \ref{geocoding_historical} foram mal calculados.

Devido ao tamanho do dataset das arvores o script de python utilizado dividiu os anos em chunks e com multiprocessamento foi calculado as especies de arvores perto do fogo.


\subsection{Locations in the middle of the sea.}
Between 2013 and 2023, some of the featured locations were in the middle of the ocean. Although having a real-life location set explicitly in the file when using services like Google Maps, its coordinates were undeniably wrong. These multiple geolocation errors were discovered when trying to pin multiple species of trees \ref{tree_species_wildfires} to a single location with a distance function calculator. The algorithm yielded values that were outside of the range spectrum of 1500km. Leading to the manual confirmation of these errors with the help of the Google Maps service.



\subsection{Creating Areas without wildfires}

Random Coordinates that were not equal to the ones in Natural Fires. A minimum distance of 25 km for each generated coordinated and the ones in Natural fires. With openstreet map given the range for latitude (36.7, 42.3) and longitude (-9.8, -6) check if the generated coordinate is in Portugal and check if it belongs to a county to ensure it isn't in the middle of the sea.

\subsection{Dataset description}
2001: 25982
2002: 25650
2003: 25138
2004: 21189
2005: 34578
2006: 19175
2007: 15615
2008: 9905
2009: 17399
2010: 14431
2011: 14913
2012: 11841
2013: 11899
2014: 3833
2015: 8431
2016: 7782
2017: 10412
2018: 5559
2019: 4040
2020: 3953
2021: 2610
2022: 4040
2023: 2499
TOTAL : 300874

Natural Fires
2001: 44
2002: 13
2003: 96
2004: 16
2005: 3
2006: 67
2007: 50
2008: 28
2009: 106
2010: 138
2011: 102
2012: 56
2013: 77
2014: 38
2015: 138
2016: 67
2017: 104
2018: 114
2019: 128
2020: 95
2021: 103
2022: 115
2023: 72
TOTAL: 1770

Reignition Fires - Averiguar os zeros
2001: 0
2002: 0
2003: 0
2004: 0
2005: 0
2006: 0
2007: 0
2008: 0
2009: 0
2010: 0
2011: 0
2012: 2256
2013: 2430
2014: 305
2015: 1505
2016: 1347
2017: 1714
2018: 712
2019: 580
2020: 524
2021: 201
2022: 480
2023: 247
TOTAL: 12301

Unknown Fires
2001: 215
2002: 221
2003: 234
2004: 268
2005: 377
2006: 1557
2007: 3577
2008: 3253
2009: 4422
2010: 6071
2011: 5915
2012: 4041
2013: 4133
2014: 2273
2015: 3834
2016: 3497
2017: 5266
2018: 2949
2019: 2506
2020: 2480
2021: 2101
2022: 3154
2023: 1863
TOTAL: 64207

NA Fires


\section{Materials and Methods}

\subsection{Study Area}

%\begin{comment}
%\end{comment}






\section{Python libraries used in the conception of the dataset}
requests
pandas
os to check if files already existed.
numpy
sklearn



\section{Entry Selection}
Specifity how many entries raw files have.
